\hypertarget{usage}{%
\section{Usage}\label{usage}}

This package takes advantage of non-equilibrium candidate Monte Carlo
moves (NCMC) to help sample between different ligand binding modes using
the OpenMM simulation package. One goal for this package is to allow for
easy additions of other moves of interest, which will be covered below.

The integrator from \textbf{BLUES} contains the framework necessary for
NCMC. Specifically, the integrator class calculates the work done during
a NCMC move. It also controls the lambda scaling of parameters. The
integrator that BLUES uses inherits from
\texttt{openmmtools.integrators.AlchemicalExternalLangevinIntegrator} to
keep track of the work done outside integration steps, allowing Monte
Carlo (MC) moves to be incorporated together with the NCMC thermodynamic
perturbation protocol. Currently, the \texttt{openmmtools.alchemy}
package is used to generate the lambda parameters for the ligand,
allowing alchemical modification of the sterics and electrostatics of
the system.

The \textbf{BLUESSampler} class in \texttt{ncmc.py} serves as a wrapper
for running NCMC+MD simulations. To run the hybrid simulation, the
\textbf{BLUESSampler} class requires defining two moves for running the
(1) MD simulation and (2) the NCMC protcol. These moves are defined in
the \texttt{ncmc.py} module. A simple example is provided below.

\hypertarget{example}{%
\subsection*{Example}\label{example}}

Using the BLUES framework requires the use of a
\textbf{ThermodynamicState} and \textbf{SamplerState} from
\texttt{openmmtools} which we import from \texttt{openmmtools.states}:

\begin{minted}[breaklines]{python}
from openmmtools.states import ThermodynamicState, SamplerState
from openmmtools.testsystems import TolueneVacuum
from blues.ncmc import *
from simtk import unit
\end{minted}

Create the states for a toluene molecule in vacuum.

\begin{minted}[breaklines]{python}
tol = TolueneVacuum()
thermodynamic_state = ThermodynamicState(tol.system, temperature=300*unit.kelvin)
sampler_state = SamplerState(positions=tol.positions)
\end{minted}

Define our langevin dynamics move for the MD simulation portion and then
our NCMC move which performs a random rotation. Here, we use a
customized \texttt{openmmtools.mcmc.LangevinDynamicsMove} which allows
us to store information from the MD simulation portion.

\begin{minted}[breaklines]{python}
dynamics_move = ReportLangevinDynamicsMove(n_steps=10)
ncmc_move = RandomLigandRotationMove(n_steps=10, atom_subset=list(range(15)))
\end{minted}

Provide the \textbf{BLUESSampler} class with an \texttt{openmm.Topology}
and these objects to run the NCMC+MD simulation.

\begin{minted}[breaklines]{python}
sampler = BLUESSampler(thermodynamic_state=thermodynamic_state,
                       sampler_state=sampler_state,
                       dynamics_move=dynamics_move,
                       ncmc_move=ncmc_move,
                       topology=tol.topology)
sampler.run(n_iterations=1)
\end{minted}