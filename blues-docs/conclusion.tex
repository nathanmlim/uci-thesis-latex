\section{Conclusion}
On the BLUES project, my primary contribution was to incorporate the best practices in software development such as: (1) version control, (2) testing and code coverage, (3) continuous integration, (4) automated build systems, (5) standardized code style, and (6) documentation.
Incorporation of these design principles ensure long-term reliability, reproducibility, and viability in the code and go towards helping others continue future work and extensions on the BLUES project.
The BLUES software package is available on Github with 80\% of the code base tested, continually integrated with Travis-CI, and the documentation available on ReadTheDocks. 
I have written the code to adhere to the PEP8 format \cite{pep8} and the documentation follows the numpydoc \cite{numpydoc} format.

Furthermore, by designing the BLUES toolkit to be as modular as possible, this has enabled easy implementation of new NCMC move types to enhance sampling beyond simple ligand rotations.
Recent work in the Mobley lab has included implementation of NCMC moves such as: sidechain rotations \cite{burley2019enhancing} for enhancing protein motions, rotations in ligand torsional angles, and `water-hopping' for enhanced sampling of water motions.
These recent studies utilize the same code base as seen in our studies with simple random ligand rotations, but have simply replaced the ligand rotation function with these alternative moves.
Since version 0.2.5, the BLUES code has been re-designed to be compatibility with other software packages like `openmmtools' \cite{openmmtools} to allow simple conversion of their Markov-chain Monte Carlo moves into NCMC moves, to be used in the NCMC+MD hybrid simulation framework that the BLUES toolkit provides.

Incorporation of these design principles ensure long-term reliability, reproducibility, and viability in the code and go towards helping others continue future work and extensions on the BLUES project. 